\begin{thanks}
	
在5年的博士生涯的科研探索中,非常感谢很多人热心、真诚的帮助与鼓励。

在学术上,对我帮助和影响最大的莫过于我的导师何赛灵教授。感谢何老师在科研上对我的细心指导和支持,给予我完善的实验条件和充满挑战性的研究课题。何老师思维敏捷,治学严谨,学识渊博,独具慧眼,让我影响深刻,值得我一生学习。

感谢浙江大学的老师和同学平时对我学术上的指导,生活上的帮助,实验上的分享,和测试上的协助。感谢你们使我的博士生涯的生活变的丰富多彩。他们是:时尧成老师,戴道锌老师,闫春生老师,马云贵老师,钱骏老师,杨柳老师,高士明老师,沈建其老师,金毅老师,胡骏老师,唐涌波师兄,刘清坤师兄,徐培鹏师兄,管小伟师兄,陈朋鑫师兄,何应然师兄,唐建伟师兄,陈拓师兄,王健师兄,孙耀然师兄,蔡涛师兄,葛萧尘师兄,付鑫师姐,鲍芳琳,王少伟,刘一超,陈思涛,于龙海,黄凌晨,王晓坤,刘珂鑫,刘琦,张磊,张建豪,马珂奇,张森林,张宇光,寇鹏飞,王楠,刘鹏浩,郑佳久,吴昊,黄圣哲,杨林,吴田甜,兰璐,马兆峰,刘旋,钟卫佳,曾锐玺,童严,王凯,孙非,古晶,郭凯凯,林宏泽,董泳江,王俊楠,魏一振,叶高翱,初立亮,彭伟,尹延龙,刘卫喜,单海丰,李晨蕾,李江,陈敬业,陈讯等同学。另外,感谢超净室实验员胡鑫松和陈辉,对我实验上的帮助和指导。

感谢华南师范大学的老师和同学在科研上的帮助和指导。感谢你们让我在博士期间感受到了不同研究小组的气氛。非常珍惜和你们在科研项目上的合作。他们是:刘柳老师,陈建新,张晨昭,朱云涛,陈楷旋等同学。

感谢根特大学光子学研究小组的老师和同学,当我在比利时时,对我科研,实验和生活上无私的帮助和指导。感谢你们让我在博士期间体验了异国风味,给我带来生活和科研上的乐趣,并且给予我崭新的科研视角。他们是:Dries Van Thourhout, Gunther Roelkens, Steven Verstuyft, Liesbet Van Landschoot, 王哲超,厉彦璐,叶楠,谢卫强,朱云鹏,王瑞军,胡琛,张菁,田斌,赵浩澜,李昂,邢宇飞,李连艳,胡应涛,陈红涛等。

感谢浙江大学在我博士期间提供多次奖学金和交流平台,感谢光电系对我专业知识的培养,感谢光及电磁波研究中心对我科研生活的帮助,感谢国家留学基金委对我的资助。

最后,最衷心地感谢武英晨对我学术和生活的指导、帮助、支持和鼓励。最衷心地感谢我的父母、奶奶、外婆和亲人对我生活的无条件支持和帮助。最衷心地感谢我在天国的爷爷和外公,对我幼年的教导。

\begin{flushright}
	黄强盛
	
	2016年4月于启真湖畔
\end{flushright}
\end{thanks}
