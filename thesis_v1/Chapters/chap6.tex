\chapter{总结与展望}
本论文主要研究了两种重要的硅基光调制器:基于硅基混合集成III-V平台上的电吸收光调制器和纯硅基上结合可调反射镜和微环的光调制器。

本论文首先回顾了目前在硅基平台上6种光调制器的实现方式。其中纯硅基光调制器和硅基外延锗硅光调制器,由于加工工艺与标准的半导体工艺完美兼容,受到研究人员的重点关注。近些年不同结构的调制器蜂拥而出,它们在速度,尺寸,功耗上都各有千秋。虽然这两种类型的光调制器是目前主流的光调制器,但也受到在硅基平台上结合新材料的光调制器的挑战。首当其冲的是硅基石墨烯光调制器。虽然,石墨烯光调制器具有光学带宽大,尺寸小的优点。但是目前其速度受到器件寄生电容的影响,无法实现高速调制。而硅基聚合物光调制器,作为目前硅基光调制器中最快速度的保持着,并且结合表面等离子体波导结构,最近受到越来越多的关注。而通过在硅基上结合其他电光材料的调制器,比如BaTiO\SB{3},LiNbO\SB{3},它们的性能差强人意。而硅基混合集成III-V光调制器,借助于成熟的III-V光调制器经验,其调制性能与纯硅基和硅基外延锗硅光调制器不分上下。我们选择了硅基混合集成III-V光调制器和纯硅基光调制器作为研究方向,并且在这两个方向上都有新的设计和发现。

对于设计硅基混合集成III-V的电吸收光调制器,我们从三个方面进行了研究设计。第一方面的研究是设计III-V多量子阱结构,我们自己编写了用于设计了III-V多量子阱中结构的程序,并且利用快速退火算法解决了设计III-V多量子阱结构中的多参数优化问题,并设计了偏振不敏感的量子阱结构。第二个方面的研究是设计混合集成III-V波导结构。我们分析了对于硅基混合集成III-V平台,矩形波导结构和蘑菇型波导结构在光学和微波上的优缺点。在相同多量子阱的尺寸下,蘑菇型波导的特征阻抗偏小,而矩形波导结构的特征阻抗能够实现达到50 $\Omega$,与标准阻抗匹配,其电光调制的3~dB带宽能达到100~GHz。第三个方面的研究是设计混合集成III-V波导和硅波导的耦合结构。利用III-V材料可被选择性腐蚀的特点,我们设计了三段式锥形耦合结构,使III-V波导在三维上缓慢变化,避免了耦合过程过程中激发出III-V波导的高阶模式,从而实现了长度只有8~$\mu m$,基模耦合效率达到95\%的耦合结构。我们还首次探索了利用电子束光刻定义和套刻这种三段式锥形结构的工艺流程。为了解决电子束在高度落差大的波导附近套刻会有尺寸展宽的问题,我们巧妙地利用了正胶的特性,避免了电子束与波导侧壁的接触,从而解决了这个问题。我们初步实现了利用电子束光刻机在硅基混合集成III-V平台上加工三段式锥形耦合结构。

借助于之前的设计思路,我们首次在硅基混合集成III-V平台上实现了基于能带填充原理的电吸收光调制器。此电吸收光调制器是采用接触式光刻定义图形。因此,根据光刻精度,我们采用了蘑菇型波导的结构,以及长度更长,工艺容差更大的两段式锥形耦合结构。我们利用III-V选择性腐蚀的特点,摸索出全湿法制作III-V波导的工艺流程,减少了工艺步骤。我们首次仿真了结合QCSE和能带填充效应的激子吸收峰随外界电压变化的漂移图。仿真结果与我们调制器的静态测试结果相吻合。基于能带填充效应的电吸收光调制器的动态性能为,驱动电压50~mV,对应动态消光6.3~dB,而速度为有1.25~Gbps。此调制器是目前所有硅基光调制器中,驱动电压最低的。而调制速度目前受到载流子寿命的限制,我们可以通过在量子阱中掺杂,使调制器的工作点在反向偏压区域,从而降低载流子寿命,提高调制速度。

电吸收光调制器作为光探测器的静态和动态性能,我们也进行了详细的测试。我们实验上验证了电吸收调制器本身就可以用作高速光探测器。其在-2~V偏压下静态响应度为0.86~A/W,且探测速度可以达到20~Gbps。最后,我们在单个硅片上集成级联的阵列波导光栅,以及混合集成III-V光调制器和光探测器,展示了单片个硅片上的光收发模块。虽然,级联阵列波导光栅的插入损耗达到了24~dB,但是基于能带填充效应的电吸收光调制器具有高消光比的特性,使调制的信号的信噪比得到提高,克服了损耗,让我们在光探测器端观测到了睁开的眼图。

对于设计纯硅基光调制器,我们首次设计了利用可调反射镜与微环相结合的光调制器。我们用时域模式耦合理论和频域的传输矩阵法,分析了微环内部的反射,对微环透射谱的影响。对于低损耗,并且工作在临界耦合点的微环,内部微小的反射将会使谐振波长的透过率发生显著变化。我们在实验上,通过在微环中引入光栅结构,验证了微环中反射对透射谱的影响。利用微环的这个性质,我们通过可调反射镜和微环相互结合的结构就能实现光强度的调制器。我们设计的硅基可调反射镜,具有大光学带宽,大工艺容差的特点。由于微环对内部发射十分敏感,因此,硅基反射镜的调制区域长度只需要20~$\mu m$。我们比较了调制区域在交趾pn掺杂和水平侧向pn掺杂的两种掺杂方式下,调制的电光调制带宽,驱动电压和动态能耗。如果使用集总电极,并且采用水平侧向pn掺杂模式,硅基可调反射镜的微环光调制器的3~dB电光调制带宽将达到67~GHz,动态能耗有2.6~fJ/bit,驱动电压能为1.22~V。

未来,对于硅基混合集成III-V电吸收光调制器和纯硅基可调反射镜的微环光调制器,可以在以下方面进行深入的研究和应用:
\begin{enumerate}[(1)]
	\item 利用硅基混合集成III-V平台上的矩形波导结构,进一步提高基于QCSE的电吸收光调制器的调制速度。通过电子束光刻定义图形,并且改善干法刻蚀InP的配方,制作矩形波导结构的硅基混合集成III-V电吸收光调制器,以使调制区域的电极阻抗达到$50~\Omega$。这样能够降低由于阻抗不匹配导致的微波反射,从而提高调制带宽。
	\item 利用混合集成技术,在硅片上同时集成用于激光器和半导体放大器的III-V外延片和用于光调制器和光探测器的III-V外延片。从而在单个硅片上实现复杂的光电子芯片,比如高速光收发模块,光电光的微波发生器。
	\item 提高基于能带填充效应的电吸收光调制器的调制速度。通过对调制器的量子阱区域进行n掺杂,使光调制器工作在反向偏压,从而降低载流子寿命,提高调制速度。进而实现高速,低驱动电压,低能耗的基于能带填充效应的电吸收光调制器。
	\item 由于纯硅基上的结合可调反射镜和微环的新型光调制器,具有小尺寸,高速,低驱动电压,低能耗,大光学带宽的特点,是一种非常有潜力能和马赫曾德与微环光调制器相竞争的方案。我们将根据Imec公司提供的工艺要求,对掩膜进行细化设计,尝试在标准的CMOS的流水线上,制作出实际器件。
\end{enumerate}
