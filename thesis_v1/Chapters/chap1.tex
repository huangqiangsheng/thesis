\chapter{绪论}
自从进入21世纪,互联网已经不仅成为人类不可分割的一部分,也成为越来越多设备不可缺少的功能。人类和设备对上网带宽的需求越来越大,这促进了信息技术领域的高速发展。为了满足人类和设备日益增长的需求,光通信已经从主干网逐渐渗入到了房内。而在不远的将未来,光通信将迈向最后一步进入到处理器内部。而这对光通信的器件设备提出了新的要求。


传统光器件,虽然性能满足要求,但是由于其价格高,尺寸大,功耗大将无法满足大规模的应用。因此,研究人员从各方面不断尝试新的材料,新的结构探索高速,小尺寸,小功耗,价格低廉的解决方案。目前这个研究领域依旧热火朝天的进行着。

本章首先将着重介绍最有希望帮助光通信迈向最后一步的硅基光电子技术,接着讨论其目前可能遇到的挑战,最后将介绍在这个领域中首次由本作者完成的工作。


\section{硅光的现状}

\section{硅光的挑战}

\section{论文的内容和创新点}

\subsection{论文内容}
\subsection{论文创新点}


