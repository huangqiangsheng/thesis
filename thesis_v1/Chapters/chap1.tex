\chapter{绪论}
自从进入21世纪,互联网已经不仅成为人类不可分割的一部分,也成为越来越多设备不可缺少的功能。人类和设备对上网带宽的需求越来越大,这促进了信息技术领域的高速发展。为了满足人类和设备日益增长的带宽需求,光通信已经从主干网逐渐渗入到了房内。而在不远的将未来,光通信将迈向最后一步进入到处理器内部。而这对光通信的器件设备提出了新的要求。


传统光器件,虽然性能满足要求,但是由于其价格高,尺寸大,功耗大将无法满足大规模的应用。因此,研究人员从各方面不断尝试新的材料,新的结构探索高速,小尺寸,小功耗,价格低廉的解决方案。目前这个研究领域依旧热火朝天的进行着。


本章首先将介绍最有希望帮助光通信迈向最后一步的硅基光电子集成技术,接着着重讨论硅基光电子器件中的硅基光调制器,介绍其目前国内外的发展现状,最后将介绍在光调制器领域内由本作者首次完成的工作。


\section{硅基光电子集成技术的发展与现状}
随着信息技术的发展,短距离通信的速率不断提高。当数据的通信速率达到10 Gbps以上时,利用金属的电互联技术将会遇到能耗,串扰,损耗和电磁干扰等问题。尤其,面对当前云计算服务器间和多核处理器内,数据的交互需要在有限的空间内同时满足大带宽、低能耗和低成本的困境时,电互联的瓶颈凸显出来。电互联的这些缺点,可以通过光通信技术来解决。然而传统光通信由于单个光器件的成本高,集成度低,阻碍了传统光通信技术在短距离通信中的应用。

\begin{figure}[htb]
	\centering
	%\includegraphics[width=\textwidth]{./Pictures/figure1.jpg}
	\includegraphics[width=12cm]{./Pictures/figure1.jpg}
	\caption{在不同通信距离下,电互联、硅基光通信和光纤通信的速率使用范围 \cite{Zuffada2012}}
	\label{figure1}
\end{figure}

硅基平台(Silicon-On-Insulator),即由硅衬底、二氧化硅绝缘层和硅薄膜构成的平台,不仅在传统半导体电子领域中有广泛运用,在微纳光子系统中也被广泛采用。硅基平台也成为了实现光电子集成芯片的理想平台。虽然,过去基于III-V材料的InP平台光电子平台已经实现了复杂的通信系统\cite{Meint2014an},比如片上波分复用的光收发器和多波长路由器,但是大规模应用需要价格低廉。除此之外,InP平台上的波导,垂直方向折射率差小只有1\%,导致波导的宽度和高度尺寸在工作波长量级。而硅基光波导在水平和高度方向都有将近60\%的高折射率差,因此光波导尺寸小。又因为硅基平台晶片的单位面积成本比InP晶片低,尺寸又比InP晶片大,所以硅基平台单个光器件的成本远小于InP平台。此外,硅基平台的小尺寸波导传输损耗最低达到 0.4 dB/cm \cite{Tsuyoshi2016low} 小于InP波导的最低传输损耗约为 1 dB/cm \cite{Meint2014an}。从而,硅基光电子平台越来在光通信领域受到人们的关注。

硅基光通信模块作为硅基光电子集成芯片的一个重要应用方向,其具有带宽大,功耗低,成本低的特点。图 \ref{figure1} 描述了电互联、硅基光通信和传统光纤通信和在不同距离下的适用速率范围 \cite{Zuffada2012}。图 \ref{figure1} 也预测了随着通信速率的逐年提高,硅基光通信在短距离将逐渐代替电互联。因此,硅基光集成芯片越来越受到各国的关注。美国在2004年率先提出了EPIC (Electronic and Photonic Integrated Circuits on Si)计划,研究硅基光电子集成平台,这将有助于通信,传感,微波光子学等研究方向的发展\cite{Shah2005}。欧洲在2008年提出了HELIOS (pHotonics ELectronics functional Integration on CMOS)计划,研究基于CMOS工艺的硅基光电子平台 \cite{HELIOS}。日本也紧跟而上,在2010年提出了PECST(Photonics and Electronics Convergence System Technology),推动硅基光电子平台的发展,实现芯片间的通信带宽密度达到10 Tb/s/cm\SP{2}\cite{Arakawa2013Silicon}。

光集成芯片的概念最早是由美国贝尔实验室的Miller在1969年提出来的\cite{miller1969}。随后1993, 美国空军科学研究实验室的Richard A. Soref提出了的硅基光电子集成芯片的概念\cite{Soref1993},见图 \ref{figure2} (a). 硅基光电子芯片是在同一片硅衬底上集成了负责逻辑和驱动的晶体管,负责光通信的激光器,调制器,光放大器,光探测器,光无源结构,光波导和光纤的耦合结构。在接下来的20多年内,全世界的知名高校和半导体公司都投入大量资金到这个领域中。在2011年,见图 \ref{figure2} (b), Intel厚积薄发推出了世界上第一款硅基光通信芯片包含了片上的激光器,纯硅调制器和硅锗探测器的光通信模块,实现了单通道12.5 Gbps的传输速率 \cite{Paniccia2011}。在2012年,见图 \ref{figure2} (c, d), IBM紧接发布了利用改进的90 nm CMOS工艺线,实现了在单个硅片上同时集成晶体管和的25 Gbps的光调制器和探测器 \cite{Assefa2012}。Intel虽然集成了激光器但是没能在单片上同时集成晶体管,而IBM虽然集成了晶体管,却没能集成激光器并且缺少完整光收发链路的展示。在2015年,美国伯克利大学和麻省理工大学的Chen Sun等首次展示了直接在商业化的45 nm CMOS流水线上,制作硅基光电子集成芯片 \cite{sun2015single}。该单块芯片,见图\ref{figure3} (a), 上不仅包含处理器,内存,还包含光收发模块。并且他们还展示了如图 \ref{figure3} (b) 所示的处理器的芯片和内存的芯片直接用光互联技术进行时时数据的运算和处理。虽然该硅基光电子芯片依旧缺少片上的激光器和放大器,但是该芯片是目前最复杂的单片硅基光电子芯片,包含了700万个晶体管和850个光模块。

\begin{figure}[htb]
	\centering
	\includegraphics[width=12cm]{./Pictures/figure2.jpg}
	\caption{ (a) 最早的硅基光电子芯片概念图 \cite{Soref1993};(b)Intel的硅基光收发模块\cite{Paniccia2011};(c,d) IBM的硅基探测器和调制器\cite{Assefa2012}}
	\label{figure2}
\end{figure}

\begin{figure}[htb]
	\centering
	\includegraphics[width=10cm]{./Pictures/figure3.jpg}
	\caption{ (a) 单片硅基光电子芯片,包含处理器,内存,光收发模块\cite{sun2015single};(b) 处理器芯片和内存芯片间光互联示意图}
	\label{figure3}
\end{figure}

\section{硅基光调制器}
硅基光调制器做为硅基光通信模块中不可缺少的一环,用于电信号向光信号的转化,影响着光通信模块的通信速度和能耗,一直硅基光通信领域的重点和难点。虽然硅基光调制器的光学结构,电极和材料可以有很多种,但是评价硅基光调制器的性能好坏可以由以下5个指标看出:
\begin{enumerate}[(1)]
	\item 调制速率(Data Transfer Rate, BR),通常用单位时间内多少比特的信息从电信号转化为光信号来表示,单位是bps。在光调制器中调制速率一般是Gbps的数量级。描述调制器的速率还可以通过调制带宽的3dB来表述($f$\SB{3dB})。数字信号光调制器的电光响应频谱如同一个低通滤波器,$f$\SB{3dB}就是调制器的电光响应降低到最大值的一半时的频率。调制器的调制速率大于调制器的调制带宽,两者的关系可以通过香农有噪信道编码定理(Shannon–Hartley theorem)联系。当没有噪声时,对于二进制编码调制速率可以达到调制带宽的两倍。
	\item 消光比(Extinction Ratio, ER)也称作调制深度,表示光在信号0和1时强度的比值,用dB表示。 消光比越高,信噪比也越好。
	\item 能耗(Energy Consumption, EC)表示将单个比特的信息从电信号转化到光信号上所消耗的电的能量。能耗的公式如下所示:
		\begin{equation}
		\label{Equ:EC}
		EC = CV_{pp}^{2}/4 + I_{average}V_{bias}/BR
		\end{equation}
	该公式分为两部分,第一部分是动态调制的能耗和第二部分是静态偏执的能耗。其中$C$是调制区域的等效电容,$V_{pp}$是驱动电压的峰峰值,$I_{average}$是直流偏置和吸收光子的而产生电流,$V_{bias}$是偏置电压,BR是调制速率。
	\item 插入损耗(Insertion Loss,IL)是调制器在全通状态下的输入光信号和输出信号的差值。插入损耗也会影响产生信号的信噪比。这些损耗包含调制器引入的反射,材料的损耗,模式不匹配引起的损耗。
	\item 器件尺寸(Footprint)是硅基光调制器光学结构加上调制区域电极结构的总尺寸。调制器的尺寸也是在硅基光调制器比较重要的参数,影响了调制器的集成度。尺寸越小的硅基光调制器在硅基光通信模块中越受到欢迎。
	\item 光学带宽(Optical bandwidth)是硅基光调制器调制波长的范围。通常谐振型的光调制器,调制波长窄,只能在谐振波长处调制。而其他类型的光调制器有比较大的光学带宽。	
\end{enumerate}

下面将讨论目前硅基光调制器在光学结构和电学结构的研究成果,概括不同材料在硅基平台上实现光调制器的最新进展。
\subsection{硅基光调制器的光学结构}
硅基光调制器如同传统的光调制器,是利用电信号改变材料折射率的实部或者虚部,从而调制器光信号的相位或者幅度。材料的折射率可以用可以表示为:
\begin{equation}
	\label{Equ:index}
	\widetilde{n(\lambda)} = n(\lambda) + j\kappa(\lambda) =  n(\lambda) + j\frac{\lambda\alpha(\lambda)}{2\pi}
\end{equation}
其中$n$为材料折射率的实部,$\kappa$为材料折射率的虚部(称作消光系数)。$\alpha$是材料单位长度的损耗。$n, \kappa, \alpha$都是和波长相关。其中$\kappa$和$\alpha$是线性关系。$n$和$\kappa$可以用通过Kramers-Kronig联系到一起\cite{o1981kramers}。

\begin{figure}[htb]
	\centering
	\includegraphics[width=12cm]{./Pictures/fig_mod_opt_type.jpg}
	\caption{ (a, b) 分别是通过调制折射率实部的谐振和非谐振型光调制器;(c, d)分别是通过调制材料损耗的谐振和非谐振型光调制}
	\label{fig_mod_opt_type}
\end{figure}
图 \ref{fig_mod_opt_type} 概括了硅基光调制器的光学结构的四大类型。前两类都是利用电光效应改变光的相位,再利用光学结构转化为光强度的变化。后两类是利用电吸收效应直接改变光的强度。

第一类如图\ref{fig_mod_opt_type} (a) 所示,通过调制折射率实部,改变谐振波长的位置,实现特定波长的调制。这类调制器的特点是结构紧凑,速度快,驱动电压小,能耗低。图\ref{fig_mod_opt_type_real} (a) 展示了最早利用这种结构实现的小尺寸硅基光调制器的实例\cite{xu2005micrometre}。该调制器利用载流子注入效应,调制硅波导的折射率,移动微环的谐振峰,从而实现谐振波长处光强的调制。

第二类如图\ref{fig_mod_opt_type} (b) 所示,也是通过调制折射率实部,实现相位的调制。这类调制器但是利用马赫曾德(MZI)结构,将一根波导上的相位变化,转变成强度调制器。其特点是速度快,光学带宽大。图\ref{fig_mod_opt_type_real} (b) 展示了最早利用这种结构的实现速度达到Gbps的硅基光调制器实例\cite{liu2004high}。该调制器也是利用载流子注入效应,实现马赫曾德一臂的相位变化,从而调制输出的光强。

第三类如图\ref{fig_mod_opt_type} (c)所示,通过调制微环的损耗,从而影响微环的谐振波长处的临界耦合系数,从而调制谐振波长的强度。这类调制器的特点是结构紧凑,驱动电压低,能耗低,速度快。图\ref{fig_mod_opt_type_real} (c) 展示了最早利用这种结构的硅基光调制器的实施方案\cite{Midrio2012graphene}。该调制器利用微环中部分硅波导上的石墨烯,调制石墨烯的损耗,导致微环损耗的变化,从而调制谐振波长处光的强度。由于这类调制器是最近2012年才提出的,目前只在氮化硅平台上有实例\cite{phare2015graphene},在硅基平台上没有实例,只有理论分析的结果\cite{Midrio2012graphene}。

第四类如图\ref{fig_mod_opt_type} (d)所示,直接通过改变波导的损耗,实现光强度的调制。这类调制器的特点是结构紧凑,能耗低,速度快,光学带宽大。图\ref{fig_mod_opt_type_real} (d) 展示了利用这种结构的硅基光调制器实例\cite{tang201150}。该调制器是将InP混合集成到硅波导上,再利用InP多量子阱材料(Multiple Quantum Well, MQW)中量子束缚Stark效应(Quantum Confined Stark Effect, QCSE)实现在不同电压下,材料吸收峰的移动,导致波导的损耗的变化,从而实现光强度的变化。

这四类调制器中的马赫曾德结构除了能用于将光的相位变化转变成强度变化外,还经常被用于高级调制码型的发生器,比如文献[\citenum{Dong2012coherent}]中就利用马赫曾德结构实现正交相移键控(Quadrature Phase-Shift Keying, QPSK)调制码形,并且结合了偏振复用,使单通道的调制速度提高到112 Gbps。
\begin{figure}[htb]
	\centering
	\includegraphics[width=12cm]{./Pictures/fig_mod_opt_type_real.jpg}
	\caption{ (a, b) 分别是通过调制折射率实部的谐振和非谐振型光调制器的实例\cite{xu2005micrometre,liu2004high};(c, d)分别是通过调制材料损耗的谐振和非谐振型光调制的实例\cite{Midrio2012graphene,tang201150}}
	\label{fig_mod_opt_type_real}
\end{figure}

\subsection{硅基光调制器的电极结构}
硅基光调制器的电极结构对调制速率和能耗也有很大影响。图\ref{fig_mod_ele_type}展示了目前硅基光调制器所有的四种电极结构。这四种电极结构已经在InP光调制器和LiNbO\SB{3}光调制器上被广泛应用。下面将讨论这四种电极结构特点。

\begin{figure}[htb]
	\centering
	\includegraphics[width=12cm]{./Pictures/fig_mod_ele_type.jpg}
	\caption{ (a) 集总电极;(b)行波电极;(c)分段传输线电极;(d)电容负载行波电极}
	\label{fig_mod_ele_type}
\end{figure}

第一种电极称为集总电极(Lumped Electrode, LE)如图\ref{fig_mod_ele_type} (a)所示, 这种电极在调制区域的长度小于100 $\mu m$的硅基光调制器中被广泛使用,尤其是小型的谐振型硅基光调制器。由于这类电极末端没有负载是开路,导致微波信号末端反射,集总电极上成为驻波,从而降低所需外界的调制电压,同时也避免了偏置电压在负载上损失的能耗。因此,集总电极在小尺寸,低驱动电压,低能耗的电极上有广泛的应用。图\ref{fig_mod_ele_type_real} (a)展示了集总电极用于InP混合集成到硅波导上的调制器\cite{tang2012energy},实现了低能耗,低驱动电压,小尺寸的调制器。

第二中电极称为行波电极(Traveling Wave Electrode, TWE)如图\ref{fig_mod_ele_type} (b)所示,这种电极在调制区域大于100 $\mu m$的硅基调制器中被广泛使用,尤其是马赫曾德的硅基光调制器。由于这行波电极调制器需要将电极考虑成为传输线,因此需要设计电极本征阻抗与标准的微波器件的本征阻抗50 $\Omega$相匹配,并且使传输线的传播常数和波导中光的传播常数相匹配,从而尽可能提高调制器的调制带宽。因此,使用行波电极的调制器调制速度高于集总电极的调制器,但是在尺寸,驱动电压,能耗上差于集总电极的调制器。图\ref{fig_mod_ele_type_real} (b)展示了行波电极用于InP混合集成到硅波导上的调制器\cite{tang2012energy},实现了高速的硅基光调制器。

第三种电极称为分段传输线电极(Segmented Transmission Line Electrode, STLE)如图\ref{fig_mod_ele_type} (c)所示, 这种电极也是主要用于调制长度大于100 $\mu m$以上,并且电极的本征电阻和50 $\Omega$相差大,或者微波和光波的传播常数相差较大的硅基光调制器。分段传输线电极是行波电极的一个更为普遍的形式,相比行波电极,其调制带宽可以进一步提高,但是能耗不会有所减少,而尺寸将更进一步增大,。图\ref{fig_mod_ele_type_real} (c)展示了分段传输线电极用于InP混合集成到硅波导上的调制器\cite{tang2012energy},实现了目前调制速度最快的硅基光调制器。

第四种电极称为电容负载行波电极(Capacitively-Loaded Traveling-Wave Electrode, CLTWE)如图\ref{fig_mod_ele_type} (d)所示, 这种电极在硅基光调制器中的应用比较少,主要用于调制长度大于500 $\mu m$以上,并且电极的本征电阻和50 $\Omega$相差大,或者微波和光波的传播常数相差较大的硅基光调制器。由于电容负载行波电极整体和如同行波电极一样,但在单个周期内,调制器区域是集总电极,因此这类电极用于兼顾传输速度和低驱动电压。不过,由于在单个调制器区域还有没参与调制的电极,因此电容负载行波电极的尺寸比较大。图\ref{fig_mod_ele_type_real} (d)展示了电容负载行波电极用于InP混合集成到硅波导上的调制器\cite{tang2012energy},实现了马赫曾德的光调制器。

这四种电极各有优缺点,如果需要尺寸小,能耗低的调制器,可以采用集总电极。如果需要尽可能高的调制速度,或者调制器区域长度比较长时则可以采用行波电极或分段传输线电极结构。而电容负载行波电极是集总电极和分段传输线折中的形式,可以在这两种电极都无法满足要求的情况下使用。
\begin{figure}[htb]
	\centering
	\includegraphics[width=12cm]{./Pictures/fig_mod_ele_type_real.jpg}
	\caption{ (a) 集总电极的实例\cite{tang2012energy};(b)行波电极实例\cite{tang201150};(c)分段传输线电极实例\cite{tang2012over};(d)电容负载行波电极实例\cite{chen2011forty}}
	\label{fig_mod_ele_type_real}
\end{figure}
\subsection{纯硅基光调制器}
纯硅的光调制器可以与标准的半导体工艺完美结合,满足实际调制器小尺寸,低功耗或者高速,大光学带宽的需求,而且调制器的加工成本低。因此,纯硅硅光调制器一直是硅基光调制器中最热门的研究方向。不过由于硅是反演中心对称的晶体,导致它不具有线性电光效应,即Pockels效应,并且在通信1.3 $\mu m$ 和 1.55 $\mu m$处,二阶非线性Kerr也很弱\cite{soref1987electrooptical}。虽然硅的热光系数很大,但是热光效应作为光调制器速度太慢\cite{cocorullo1992thermo}。

目前,纯硅基光调制器采用的是等离子色散效应(plasma dispersion effect)实现速度达到Gbps以上的光调制器。硅基光调制器采用三种方式实现等离子色散效应,分别是载流子注入(Carrier injection),载流子聚集(Carrier accumulation)和载流子耗尽(Carrier depletion)。

载流子注入的实现方式是最早被提出来的,但是靠这种方式的调制器速度由硅中少数载流子寿命决定。目前载流子注入的光调制器速度最多达到Gbps,通过预加重(pre-emphasis)的特殊码形速度目前最高达到18 Gbps\cite{manipatruni2007high}。

载流子聚集对的方式是最早实现高于1 Gpbs的纯硅基光调制器。载流子聚集调制器是通过将自由载流子聚集到波导中的薄介质层附近从而实现改变折射率。因此,载流子聚集的调制器速度突破了少数载流子寿命的约束,调制速度由等效电容和电阻决定速率。由于薄介质层的厚度薄,载流子聚集效率低,等效电容大,从而影响了最高的速度,目前载流子聚集的光调制器最快达到25 Gbps\cite{fujikata201025}。

载流子耗尽的方式是目前实现最快硅基光调制器。载流子耗尽调制器是通过外界反向偏压,控制波导中耗尽区和光场相互作用的面积从而实现调制。因此,载流子耗尽的调制器也突破了载流子寿命的约束,调制速度也由等效电容和电阻决定速率。由于耗尽区的宽度可以通过本pn结的掺杂浓度,或者pin结构中i层的厚度,或者反向偏压的大小控制,因此等效电容可以方便的根据实际所需速率设计。目前载流子耗尽的光调制器最快速度达到了60 Gbps\cite{Xiao201360}。

由于纯硅基调制器和半导体工艺完美结合,因此大学实验室只能提供设计,然后提交给拥有完全流水线的半导体代工公司(比如比利时的Imec-ePIXfab \cite{Imec, absil2015silicon}、新加坡的OpSIS-IME \cite{novack201330}、美国的IBM的45nm的SOI半导体流水线\cite{narasimha2006high}等)生产制作。另外,中国的中芯国际(Semiconductor Manufacturing International Corporation, SMIC)也提供硅调制器的加工\cite{xu2012high,xiao2013high,Xiao201360}。表\ref{sil_mod}总结了近期纯硅基光调制器方面的最好的结果。目前纯硅基光调制器的最高调制速度达到了60 Gbps,能耗最低能到0.1 fJ/bit,驱动电压最低需要50 mV。

{
	\begin{table}[htb]
		\zihao{5}
		\caption{比较近期纯硅基光调制器的性能。PhC: 光子晶体;Ring:微环; Zigzag:锯齿;Disk:微盘;MZI:马赫曾德}
		\label{sil_mod}
		\centering
		\begin{tabular}[t]{p{1.5cm}ccp{1.2cm}ccccccc}
			\hline
			掺杂方式 & 光学结构 & 电极结构 & 调制区尺寸 & 调制速度 & 动态能耗 & 消光比 & 插入损耗 & 调制电压\\
			\hline
			lateral pn \cite{xiao2013high,Xiao201360} & MZI & TW  & 750 $\mu m$ & 60 Gbps & - & 4.4 dB & 2.0 dB& 6.5 V\\
			zigzag pn\cite{Xiao201360} & Ring & LE & 22 $\mu m$ & 60 Gbps & - & 4.2 dB & - & 6 V\\
			vertical pn\cite{timurdogan2014ultralow} & Disk & LE & 4.8 $\mu m$ & 44 Gbps & 17.4 fJ/bit & 8.0 dB & 0.92 dB & 2.2 V\\
			lateral pin\cite{shakoor2014ultra} & PhC & LE & ~6.0 $\mu m$ & 1 Gbps & 0.1 fJ/bit & 2.0 dB & - & 50 mV\\			
			lateral pn\cite{Pantouvaki201556} & Ring & LE & 10 $\mu m$ & 56 Gbps & 45 fJ/bit & 4.0 dB & 3.0 dB& 2.5 V\\
			lateral pipin\cite{Ziebell201240} & MZI & TW & 950 $\mu m$ & 40 Gbps & - & 3.2 dB & 2.5 dB & 6 V \\
			lateral pin\cite{Baba201350} & Ring & LE & 20 $\mu m$ & 50 Gbps & - & 4.6 dB & 5.2 dB & 1.96 V\\
			lateral pn\cite{Ding2013electro} & MZI & TW & 2 $mm$ & 40 Gbps & 32.4 fJ/bit & 3.5 dB & 12.5 dB & 0.36 V\\
			\hline
		\end{tabular}
	\end{table}
}

纯硅基光调制器波导的掺杂方式根据不同等离子色散实现方式分有很多种\cite{reed2010silicon},在此主要介绍三种常用且特色的方案,如图\ref{fig_silicon_mod_cross}所示。表\ref{sil_mod}也展示不同掺杂方式的硅基光调制器最新最好的性能。
第一种是水平掺杂的pn结或者pin结,如图\ref{fig_silicon_mod_cross}(a)所示。这种掺杂方式使用最为广泛,用于目前调制速度达到60 Gbps的纯硅基MZI光调制器(如图\ref{fig_silicon_mod}(a)所示)\cite{xiao2013high}。另外,驱动电压只有50 mV 并且能耗只有0.1 fJ/bit的光子晶体(Photonic Crystal, PhC)结构的纯硅基光调制器(如图\ref{fig_silicon_mod}(b)所示)也是采用了水平掺杂的pn结\cite{,shakoor2014ultra}。
第二种是交趾掺杂和其改进型的锯齿掺杂,如图\ref{fig_silicon_mod_cross}(b)所示。这种掺杂方式目前主要用于提高单位长度的调制效率,目前调制速度最快达到60 Gbps的纯硅基微环光调制器(如图\ref{fig_silicon_mod}(c)所示)就是采用这种结构\cite{Xiao201360}。
第三种是在2014年才提出的垂直方向pn结,如图\ref{fig_silicon_mod_cross}(c)所示。通过种掺杂方式,并且结合微盘结构(如图\ref{fig_silicon_mod_cross}(c)所示),具有pn掺杂谐振型光调制器中最高的电光效应,达到了250 pm/V,这是之前的十倍\cite{timurdogan2014ultralow}。

\begin{figure}[htb]
	\centering
	\includegraphics[width=15cm]{./Pictures/fig_silicon_mod_cross.jpg}
	\caption{ (a) 中心偏移的水平pn结\cite{xiao2013high};(b)齿状交趾的纵向pn结\cite{Xiao201360};(c)垂直方向的pn结\cite{timurdogan2014ultralow}}
	\label{fig_silicon_mod_cross}
\end{figure}

\begin{figure}[htb]
	\centering
	\includegraphics[width=15cm]{./Pictures/fig_silicon_mod.jpg}
	\caption{ (a) 60 Gbps的MZI纯硅基光调制器\cite{Xiao201360};(b)能耗0.1 fJ/bit的光子晶体纯硅基光调制器\cite{shakoor2014ultra};(c)60 Gbps的锯齿掺杂微环纯硅基光调制器\cite{Xiao201360}}
	\label{fig_silicon_mod}
\end{figure}
\subsection{硅基外延锗硅光调制器}
为了克服纯硅基是间接带隙对的半导体,没有电吸收效应的缺点,因此需要将其他材料和硅集成实现光调制器。锗材料具有直接带隙半导体基于Franz-Keldysh现象的电吸收效应\cite{frova1965franz},并且锗材料也是如今CMOS工艺的标准材料,因此硅基外延锗硅(GeSi)光调制器成为了一个热门的研究方向。最早实现GeSi光调制器是2008年\cite{liu2008waveguide},研究人员通过在硅上先生长一层仅40 nm的GeSi缓冲层,在高温生长在Ge中掺有少量Si的GeSi材料,最后高温退火减少材料中的缺陷。除此之外,研究人员虽然在2005年,实现了硅上生长GeSi多量子阱结构并且发现了多量子阱中量子束缚Stark效应(Quantum Confined Stark Effect, QCSE)\cite{kuo2005strong},但是,研究人员在2010年,才实验展示了基于硅基GeSi多量子阱结构QCSE的电吸收调制器\cite{rong2010quantum}。表\ref{sil_ge_mod}展示了目前不同GeSi材料下的硅基外延锗光调制器的最佳工作性能。
{
	\begin{table}[htb]
		\zihao{5}
		\caption{近期硅基外延锗硅光调制器的最佳性能。FK:Franz-Keldysh;}
		\label{sil_ge_mod}
		\centering
		\begin{tabular}[t]{p{1.5cm}ccp{1.2cm}ccccccc}
			\hline
			调制原理 & 工作波长 & 电极结构 & 调制区尺寸 & 调制速度 & 动态能耗 & 消光比 & 插入损耗 & 调制电压\\
			\hline
			Ge FK\cite{Srinivasan201656} & 1615 nm & LE  & 40 $\mu m$ & 56 Gbps & 12.8 fJ/bit & 4.6 dB & 4.9 dB& 2 V\\
			GeSi FK\cite{Dazeng2013high} & 1550 nm & LE  & 50 $\mu m$ & 28 Gbps & 147 fJ/bit & 5.9 dB & 4.8 dB& 3 V\\
			Ge/GeSi QCSE\cite{chaisakul201223} & 1448 nm & LE & 90 $\mu m$ & 23 GHz & 108 fJ/bit & - & - & 1V\\
			\hline
		\end{tabular}
	\end{table}
}

从表\ref{sil_ge_mod}可以看出,目前调制速度最快(56 Gbps)和能耗最低(12.8 fJ/bit)的都是基于纯锗Franz-Keldysh现象的光调制器,其结构如图\ref{fig_ge_mod}(a)所示。它的尺寸和速度都可以与纯硅基微环光调制器相媲美,并且硅基锗的光调制器光学带宽大于22nm高于纯硅基微环光调制器\cite{Srinivasan201656}。不过,纯锗的光调制器工作波长是1615 nm附件,而合理配比Ge和Si的材料的组分能将工作波长移到1550 nm。目前硅基GeSi的光调制器速度达到28 Gbps依旧有提升的空间\cite{Dazeng2013high},结构如图\ref{fig_ge_mod}(b)。另外,由于基于Ge/GeSi 多量子阱结构光调制的也能调整工作波长,但是由于这种外延结构制作最为困难,外延结构如图\ref{fig_ge_mod}(c)所示,因此发展的比较缓慢。目前它在模拟信号下最快的调制速度只有23 GHz,工作波长是1448 nm。因此,硅基外延锗的光调制器仍需要继续研究,从而实现在通信波长1.55 $\mu m$处的高速,低功耗的光调制器。除此之外,由于硅波导和锗硅调制器的材料不同,波导结构尺寸不同,因此为了实现单片集成,任需要设计硅波导到锗硅调制器的光学耦合结构。目前采用的是通过硅波导的宽度渐变的锥形结构和硅锗波导进行端面耦合,如图\ref{fig_ge_mod}(d)所示。

\begin{figure}[htb]
	\centering
	\includegraphics[width=12cm]{./Pictures/fig_ge_mod.jpg}
	\caption{ (a) 硅基锗光调制器\cite{Srinivasan201656};(b)硅基锗硅光调制器\cite{Dazeng2013high};(c)硅基锗硅多量子阱光调制器\cite{chaisakul201223};(d)硅波导和锗硅光调制器的耦合结构\cite{Dazeng2013high}}
	\label{fig_ge_mod}
\end{figure}
\subsection{硅基混合集成新型材料光调制器}
\subsection{硅基混合集成III-V光调制器}

\subsection{国内硅基光调制器的进展}
\section{论文的内容和创新点}

\subsection{论文内容}
\subsection{论文创新点}


