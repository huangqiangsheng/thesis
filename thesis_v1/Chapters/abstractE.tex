\begin{englishabstract}
Silicon-on-insulator platform is not only widely used in traditional semiconductor field, but also widely used in nanophotonic system. Silicon-on-insulator platform becomes an ideal platform to realize the optoelectronic nano chips. The optical communication module, belonging to optoelectronic chips, can be used to improve the communication speed among chips, and decrease the energy consumption in communication. The optical communication module based on silicon-on-insulator platform also bring new challenges to the design and fabrication in the traditional semiconductor. So, the optical communication module based on silicon-on-insulator platform has great practical and research values. Optical modulator is an indispensable part in the optical communication module and attracts lots of attention. Low driving voltage optical modulator with simple driving voltage and low energy consumption, attracts more and more attention in the optoelectronic integrated field. In this thesis, we focus on two new schemes to realize the low driving voltage optical modulator in silicon-on-insulator platform.

The first new scheme is based on hybrid integration technology. We can integrate the direct bandgap III-V material on the silicon waveguide. The low driving voltage optical modulator can be achieved by the band-filling effect with electroabsorption phenomenon in III-V multi-quantum wells. In this hybrid integration platform, we design a novel three section taper coupler to reduce the coupling length between the silicon waveguide and the hybrid III-V waveguide, by inhibiting the unwanted high-order modes excited in the III-V waveguide. The coupler can achieve more than 95\% coupling efficiency in 100 nm bandwidth, with only 8 $\mu m$ length. Based on this design method, we fabricate and measure a silicon based hybrid III-V electroabsorption modulator. Thanks to the highly selective wet etching between different III-V materials, we find a new way to fabrication III-V waveguide only with wet etching process, simplifying the traditional  fabrication process. Then, we firstly present a low driving voltage electroabsorption modulator based on band-filling effect. For this 80 $\mu m$ long modulator, 1.25 Gbps modulation with a 6.3 dB extinction ratio is obtained using only a 50 mV peak-to-peak driving voltage and 0.29 fJ/bit dynamic power consumption. It is one of the lowest driving voltage modulators so far. Band-filling effect based electroabsorption modulator gives us a new way to realize the low driving voltage, low energy consumption and compact footprint optical modulator.

Thanks to the dual functional  electroabsorption modulator in the reverse bias, we measured its property as a photodetector. We test and verify the electroabsorption can also be used as a high speed photodetector. In the -2 V reverse bias, the responsivity of the photodetector is 0.86 A/W, and its speed can be 20 Gbps. Based on its property, we firstly present a transceiver based on a single hybrid integrated III-V and silicon chip, including 2 cascaded arrayed waveguide gratings, 6 high speed optical modulators, 6 high speed photodetectors. Thanks to the high extinction ratio in the band-filling effect based electroabsorption modulator, we overcome the large insertion loss from the cascaded arrayed waveguide gratings. When the transmission speed in the single channel is 1.5 Gbps, we can see a clean open eye pattern from the photodetector.

In the second new scheme, we find that the low loss microring resonator is very sensitive to its internal reflection, so we design a new silicon optical modulator based on microring resonator integrated with a tunable reflector. This silicon based tunable reflector microring optical modulator not only has a larger optical bandwidth than the microring modulator, but also has a smaller footprint than the Mach-Zehnder modulator. The modulator section in this optical modulator is 200 $\mu m$ long, with a 0.5 V theoretical driving voltage. We firstly analyze the influence of the photon lifetime on the modulation bandwidth for this reflectivity modulation based modulator. We find that the modulation bandwidth is limited by the lifetime of the photons.

\englishkeywords{photonic integrated circuit, silicon based optical modulator, hybrid silicon platform, electroabsorption modulator, photodetector, optical transceiver}

\end{englishabstract}
