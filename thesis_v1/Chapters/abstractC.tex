\begin{abstract}

硅基平台不仅在传统半导体电子领域中有广泛运用,在微纳光子系统中也被广泛采用。
硅基平台成为了实现微纳光电子集成芯片的理想平台。光电子集成芯片中的光通信模块,能提高芯片间的通信速度,降低通信功耗。 
硅基光通信模块也给传统的半导体设计和制作带来新的挑战。因此,硅基光通信模块有着巨大的研究和实用价值。
光调制器做为光通信模块中不可缺少的一环,一直是该领域关注的重点。
其中低驱动电压的光调制器由于其驱动电路简单,能耗低,在光电子集成芯片中受到越来越多的关注。
本论文的研究包含了硅基平台上实现低驱动电压光调制器的两种新型方案。

第一个创新方案,采用混合集成技术,将直接带隙的III-V多量子阱材料直接键合到硅基光波导上面,利用III-V多量子阱材料的能带填充效应下的电吸收现象实现光调制器。在这种硅基混合平台上,本文创新地设计了三段锥形耦合结构,从而抑制III-V波导中高阶模的激发,缩短了纯硅波导和混合集成III-V波导之间的耦合长度。该耦合结构长度只有8 $\mu$m,就能实现95\%以上能量的耦合。凭借这种设计思路,本文制作和测试了硅基混合集成的III-V电吸收光调制器。利用III-V材料高选择性腐蚀比的特性,我们摸索出全湿法制作III-V波导的工艺,简化了传统III-V波导的制作流程。我们首次展示了基于能带填充效应的低驱动电压电吸收光调制器。该调制器的长度有80  $\mu$m,驱动电压值只有50 mV,动态消光达到6.3 dB, 动态能耗只有0.29 fJ/bit,与此同时调制速率有1.25 Gbps。 这是目前报道中驱动电压最低的光调制器之一。基于能带填充效应的电吸收光调制器提供了一种实现低驱动电压,低功耗,小尺寸光调制器的新思路。

借助于电吸收光调制器在反偏电压下具有双工作状态的特点,我们测试了其作为光探测器的性能。我们验证了电吸收光调制器也可以作为高速的光探测器。它在-2 V的偏压下有0.86 A/W的响应度,并且它的探测速度能达到20 Gbps。利用这个特点,我们首次展示了集成级联的两个阵列波导光栅,6个高速光调制器,6个高速光探测器的单片硅基混合集成的光收发模块。借助于能带填充效应下,电吸收光调制器高消光比的特点,我们克服了级联阵列波导光栅的高插入损耗。当单个信道的收发传输速率在1.5~Gbps时,我们在光探测器端观测到了清晰的睁开的眼图。 

第二个创新方案, 利用低损耗微环对内部反射敏感的特点,我们设计了新型的基于可调反射镜和微环结构的纯硅基光调制器。这种硅基可调反射镜的微环光调制器,既比微环光调制器结构有更大的光学带宽,也比马赫-曾德尔光调制器的结构紧凑。我们设计的光调制器相位调制区域有200 $\mu$m时,理论预测的驱动电压只需要0.5 V,也能实现8 dB的消光比。接着,我们首次分析了调制微环内部反射率时,微环内光子寿命对调制带宽的影响。我发现此光调制器的调制带宽受到光子寿命的限制。

\keywords{集成光路~~硅基光调制器~~硅基混合集成平台~~电吸收调制器~~光探测器~~光收发模块}
\end{abstract}
